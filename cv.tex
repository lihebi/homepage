\documentclass[10pt,letterpaper]{article}
\usepackage[parfill]{parskip}
\usepackage{array}
\usepackage{ifthen}
\usepackage{fontawesome}
\usepackage{hyperref}
\usepackage[usenames, dvipsnames]{color}
\pagestyle{empty}
\usepackage[default]{cantarell}
\usepackage[T1]{fontenc}
\usepackage[left=0.75in,top=0.6in,right=0.75in,bottom=0.6in]{geometry}
\usepackage{fancyhdr}

\newenvironment{mysection}[1]{ % 1 input argument - section name
  \medskip
  \MakeUppercase{\bf #1}
  \medskip
  \hrule
  \medskip
  \begin{list}{}{
      \setlength{\leftmargin}{1.5em}
    }
  \item[]
}{
  \end{list}
}


\definecolor{mygray}{gray}{0.6}
\begin{document}

\centerline{\MakeUppercase{\huge\bf Hebi Li}}
\medskip
%% \centerline{PhD Candidate in Programming Languages \& Analysis}
%% \centerline{\textit{\textcolor{mygray}{1865 Long Rd, Unit A, Ames IA
%%       50010}}}
\centerline{\faEnvelope ~ lihebi.com@gmail.com ~ | ~ \faHome ~
  \href{http://lihebi.com}{lihebi.com} ~ | ~ \faGithubSquare ~ lihebi
  %% ~ | ~ \faPhone +1 (515) 708-8131
}

\begin{mysection}{Education}
  \textbf{Iowa State University} \hfill \emph{Aug. 2014 - Aug 2019 (Expected)} \\
  \emph{PhD Student in Department of Computer Science}

  \textbf{University of Science and Technology of China}
  \hfill \emph{Aug. 2010 - Jun. 2014} \\
  \emph{B.E. in Electrical Engineering}
\end{mysection}


\begin{minipage}{0.5\textwidth}
  \begin{mysection}{Research Area}
    \begin{itemize}
    \item Artificial Intelligence, Causality
    \item Natural Language Processing
    \item Programming Languages \& Analysis
    \end{itemize}
  \end{mysection}
\end{minipage}
\begin{minipage}{0.5\textwidth}
  \begin{mysection}{Programming Skills}
    \begin{itemize}
    \item Lisp \& Scheme
    \item C/C++
    \item Python
    \end{itemize}
  \end{mysection}
\end{minipage}
  

\begin{mysection}{Research Projects}

  % \textbf{NLP: Semantic metric for abstract doc summarization (Python,
  %   Tensorflow)} \hfill \emph{Spring 2018 - Present}
  % \begin{itemize}
  % \item Website \& Code: \url{https://github.com/lihebi/anti-rouge}
  % \item Description: ROUGE is the de facto criterion for summarization
  %   research.  However, its two major drawbacks: 1. favors lexical
  %   similarity instead of semantic similarity 2. require a reference
  %   summary which may be expensive to obtain.  Therefore, we introduce
  %   a completely new end-to-end metric system for summary quality
  %   assessment by leveraging recently developed \textbf{deep sentence
  %     embedding}.

  %   % We develop two \textbf{negative sample generation} approaches:
  %   % \textit{random mutation} and \textit{cross-pairing}.  We apply and
  %   % evaluate three sentence-embedding models: \textbf{Universal
  %   %   Sentence Encoder} by Google and \textbf{InferSent} by Facebook,
  %   % and a vanilla word-embedding model \textbf{Grove}. We evaluate 3
  %   % neural network architectures atop the embedding,
  %   % \textbf{Fully-connected}, \textbf{CNN}, and \textbf{LSTM}.

  % % \item \textbf{Hebi Li}, Qi Xiao, Yinfei Yang, Forrest Sheng Bao,
  % %   \textit{``End-to-end semantics-based summary quality assessment
  % %     for single-document summarization''}, In submission.
  % \end{itemize}
  
  \textbf{AI: Causal Discovery From High Dimensional Data (Python,
    Tensorflow)} \hfill \emph{Spring 2018 - Present}

  \begin{itemize}
  \item Description: Causal relationships are fundamental for
    predictions of the consequences of actions. However, most
    causality research are done in low-dimensional data. We hence are
    interested in \textbf{causal discovery in high dimensional data}
    such as images and text. In particular, we are interested in
    mapping low-level raw pixels (micro variables) into high-level
    features (macro variables) that have explicit causal relations.
    The model is built atop \textbf{Variational Causal Encoder (VAE)}.
  %% \item Publications: Hebi Li, Jin Tian, \textit{``Causal Discovery
  %%   From High Dimensional Data''}, In Progress.
  \end{itemize}
  
  \textbf{PL: Demand-driven Dynamic Program Analysis (C++,
    LLVM/Clang, Racket)} \hfill \emph{Summer 2015 - Spring 2018}

  \begin{itemize}
  \item Description: We develop a framework that debugs programs
    on-demand, preventing running time overhead by running just enough
    code. It analyzes a buggy program and generates a much smaller
    partial program that retains the same bug.
  % \item Website: \url{https://helium.lihebi.com/}. V1 Code:
  %   \url{https://github.com/lihebi/Helium} (723 commits, 22k
  %   C/C++). V2 Code: \url{https://github.com/lihebi/helium2} (143
  %   commits, 1.7k Racket, 3k C/C++).
  % \item Description: We develop Helium, a framework that debugs
  %   programs on-demand, preventing running time overhead by running
  %   just enough code. It analyzes a buggy program and generates a much
  %   smaller partial program that retains the same bug.  It features a
  %   \textbf{syntactic patching algorithm} that find the extra code in
  %   addition to the user selection that is necessary for a valid
  %   partial program. It also features a \textbf{demand-driven context
  %     search algorithm} to find smaller partial programs that preserve
  %   a given program property.
  % \item Implementation highlights: The framework uses
  %   \textbf{LLVM/Clang} framework as the underlying parser, and
  %   performs \textbf{AST} modelling and manipulation, and generates
  %   partial programs together with dynamic tests. The framework is
  %   provided as \textbf{docker} image that is easy and ready to
  %   use. Partial program ASTs are visualized through \textbf{Graphviz}
  %   framework.
  %% \item \textbf{Hebi Li}, Wei Le, \textit{``Enabling Dynamic Analysis
  %%   for Partial Programs Via Syntactic Patching''}, In submission.
  %% \item \textbf{Hebi Li}, Wei Le, \textit{``Demand-Driven Dynamic
  %%   Analysis for Automatic Benchmark Building''}, In submission.
  \end{itemize}

  %% \textbf{USTC Bachelor Thesis: Named Data Network} \hfill
  %% \emph{Summer 2013 - Summer 2014}

  %% Publications: Hebi Li, Xiaobin Tan \textit{``TreeSync: A Distributed
  %%   Message Synchronization Algorithm Using Topology-Related Hierarchy
  %%   over Named Data Network''}, USTC Thesis, 2014
\end{mysection}

%% \begin{mysection}{Open Source Projects}
%%   \begin{itemize}
%%   \item \textbf{hn.el (ELisp)}: A Hacker News Client for Emacs.
%%     \url{https://github.com/lihebi/hn.el}
%%   \item \textbf{simple-drill.el (Elisp)}: A Flashcard Program for
%%     Emacs. \url{https://github.com/lihebi/simple-drill.el}
%%   \item \textbf{smart-scholar.el (Elisp, Python)}: Download and manage
%%     bibs for CS conferences.
%%     \url{https://github.com/lihebi/smart-scholar.el}
%%   \end{itemize}
%% \end{mysection}

%% \begin{mysection}{Teaching Experience}
%%   TA in Network Programming, Applications, and Research Issues (ISU CS587)
%%   \hfill \emph{Spring 2019}
  
%%   TA in Advanced Design and Analysis of Algorithms (ISU CS511)
%%   \hfill \emph{Spring 2019}
  
%%   TA in Principles of Programming Languages (ISU CS342)
%%   \hfill \emph{Fall 2016, Fall 2017}

%%   TA in Introduction to Database System (ISU CS363)
%%   \hfill \emph{Fall 2014, Spring 2015}
%% \end{mysection}

%% \begin{mysection}{Awards}
%%   \textbf{Teaching Excellent Award} \hfill \emph{May 2017}
%% \end{mysection}

\end{document}
