\documentclass[10pt,letterpaper]{article}
\usepackage[parfill]{parskip} % Remove paragraph indentation
\usepackage{array} % Required for boldface (\bf and \bfseries) tabular columns
\usepackage{ifthen} % Required for ifthenelse statements
\usepackage{fontawesome}
\usepackage{hyperref}
\usepackage[usenames, dvipsnames]{color}
\pagestyle{empty} % Suppress page numbers

\renewcommand\refname{}


%----------------------------------------------------------------------------------------
%	SECTION FORMATTING
%----------------------------------------------------------------------------------------

% Defines the rSection environment for the large sections within the CV
\newenvironment{rSection}[1]{ % 1 input argument - section name
  \sectionskip
  \MakeUppercase{\bf #1} % Section title
  \sectionlineskip
  \hrule % Horizontal line
  \begin{list}{}{ % List for each individual item in the section
      \setlength{\leftmargin}{1.5em} % Margin within the section
    }
  \item[]
}{
  \end{list}
}

%----------------------------------------------------------------------------------------
%	WORK EXPERIENCE FORMATTING
%----------------------------------------------------------------------------------------

\newenvironment{rSubsection}[4]{ % 4 input arguments - company name, year(s) employed, job title and location
  {\bf #1} \hfill {#2} % Bold company name and date on the right
  \ifthenelse{\equal{#3}{}}{}{ % If the third argument is not specified, don't print the job title and location line
    \\
      {\em #3} \hfill {\em #4} % Italic job title and location
  }\smallskip
  \begin{list}{$\cdot$}{\leftmargin=0em} % \cdot used for bullets, no indentation
    \itemsep -0.5em \vspace{-0.5em} % Compress items in list together for aesthetics
}{
  \end{list}
  \vspace{0.5em} % Some space after the list of bullet points
}

% The below commands define the whitespace after certain things in the document - they can be \smallskip, \medskip or \bigskip
\def\namesize{\huge} % Size of the name at the top of the document
\def\addressskip{\smallskip} % The space between the two address (or phone/email) lines
\def\sectionlineskip{\medskip} % The space above the horizontal line for each section 
\def\nameskip{\bigskip} % The space after your name at the top
\def\sectionskip{\medskip} % The space after the heading section


\usepackage[default]{cantarell}
\usepackage[T1]{fontenc}
\usepackage[left=0.75in,top=0.6in,right=0.75in,bottom=0.6in]{geometry} % Document margins
\usepackage{fancyhdr}

%% \name{Hebi Li}
%% \address{}
%% \address{}
%% \address{+1 (515) 708-8131}
%% \address{}


%% \address{www.lihebi.com}
\definecolor{mygray}{gray}{0.6}
\begin{document}

{
  \footnotesize
  \begingroup
  \hfil{\MakeUppercase{\namesize\bf Hebi Li}}\hfil
  \nameskip\break
  \endgroup

  %% \centerline{PhD Candidate in Programming Languages \& Analysis}
  \centerline{\textit{\textcolor{mygray}{1865 Long Rd, Unit A, Ames IA
        50010}}} \centerline{\faEnvelope ~ hebi@iastate.edu ~ | ~
    \faHome ~ lihebi.com ~ | ~ \faGithubSquare ~ lihebi} }

\begin{rSection}{Education}
  \textbf{Iowa State University} \hfill \emph{Aug. 2014 - Present}
  \\ \emph{PhD Student in Department of Computer Science}

  \textbf{University of Science and Technology of China} \hfill
  \emph{Aug. 2010 - Jun. 2014} \\ \emph{B.E. in Electrical Engineering}

\end{rSection}

%% \begin{rSection}{Skills}
%%   \textbf{Programming Languages}
%%   \begin{itemize}
%%   \item \textbf{Lisp (Scheme):} Proficient
%%   \item \textbf{C/C++:} Proficient
%%   \item \textbf{Python:} Good
%%   \item \textbf{Haskell:} OK
%%   \end{itemize}

%%   \textbf{Software \& System}
%%   \begin{itemize}
%%   \item \textbf{Compiler Architecture}: Proficient
%%   \item \textbf{Programming Language}: Proficient
%%   \item \textbf{Program Analysis}: Proficient
%%   \item \textbf{LLVM/Clang}: Proficient
%%   \item \textbf{Linux:} Proficient, 8 years experience
%%   \end{itemize}
%% \end{rSection}

%% \begin{rSection}{Research Interest}
%%   \textbf{Artificial Intelligence}
%%   \textbf{Functional Programming}
%% \end{rSection}

\begin{rSection}{Research Project}
  \textbf{Dynamic Program Analysis on Demand} \hfill \emph{Summer 2015 -
    Present}

  %% I built a compiler framework,
  %% Helium \footnote{\url{https://github.com/lihebi/helium}}
  %% \footnote{\url{https://github.com/lihebi/helium2}}. It tests
  %% arbitrary, non-continuous segment of code. The front-end is built on
  %% top of Clang, and back-end algorithm is implemented in C++ and
  %% Racket.  The code has 866 commits, 1.7k Lisp (Racket), 25k C/C++.

  %% 143 commits, 1.7k Racket, 3k C/C++
  %% 723 commits, 22k C/C++

\end{rSection}

\begin{rSection}{Teaching}
  TA in Principles of Programming Languages \hfill \emph{Fall 2016, Fall 2017}

  TA in Introduction to Database System \hfill \emph{Fall 2014, Spring
    2015}
\end{rSection}

\begin{rSection}{Awards}
  \textbf{Teaching Excellent Award} \hfill \emph{May 2017}
\end{rSection}


%% TODO machine learning class

\begin{rSection}{Publications}
  %% \textbf{Hebi Li}, Xiaobin Tan \textit{``TreeSync: A Distributed
  %%   Message Synchronization Algorithm Using Topology-Related Hierarchy
  %%   over Named Data Network''}, USTC Thesis, 2014
\end{rSection}
  \textbf{Hebi Li}, Wei Le, \textit{``Enabling Dynamic Analysis for
    Partial Programs Via Syntactic Patching''}, submitted to FSE-18.

  \textbf{Hebi Li}, Wei Le, \textit{``Demand-Driven Dynamic Analysis
    for Automatic Benchmark Building''}, full paper draft.
\end{rSection}

\end{document}
